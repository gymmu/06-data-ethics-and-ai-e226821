\documentclass{article}

\usepackage[ngerman]{babel}
\usepackage[utf8]{inputenc}
\usepackage[T1]{fontenc}
\usepackage{hyperref}
\usepackage{csquotes}


\usepackage[
    backend=biber,
    style=apa,
    sortlocale=de_DE,
    natbib=true,
    url=false,
    doi=false,
    sortcites=true,
    sorting=nyt,
    isbn=false,
    hyperref=true,
    backref=false,
    giveninits=false,
    eprint=false]{biblatex}
\addbibresource{../references/bibliography.bib}

\title{Review des Papers "Wenn die Grenzen zwischen Menschen und KI verwischen" von Lucy Wehrli}
\author{Ann-Cathrine Böller}
\date{\today}

\begin{document}
\maketitle

\abstract{
    Dies ist ein Review der Arbeit zum Thema "Wenn die Grenzen zwischen Menschen und KI vewischen" von Lucy Wehrli.
}

\section{Review}
Ich finde die Fragestellung "Wenn die Grenzen zwischen Menschen und KI verwischen" sehr gut formuliert und sehr interessant. Ausserdem ist die Einleitung sehr gut geschrieben. Besonders gut finde ich den Bezug zu unserem Alltag. Die Überleitung von der Einleitung zur Fragestellung ist auch gut gelungen. 
Bei dem Untertitel "Sind KIs wirklich intelligent?" wurde das grosse Datenrepertoire gut aufgezeigt und mit Beispielen erklärt. Jedoch fehlt mir ein wenig die Erklärung wie genau KIs funktionieren, bzw wie sie trainiert werden. Den Turing Test hast du sehr gut erklärt. Mithilfe des ELIZA-Effektes hast du die Probleme im Bezug mit dem Vermenschlichen der KI deutlich aufgezeigt, was ich sehr gut finde. Die ethischen Folgen hast du auch sehr schön beschrieben. Insbesondere finde ich den Abschluss sehr gelungen. Mit der Frage regst du den Leser/ die Leserin zum Nachdenken an. Ausserdem wird ganz deutlich aufgezeigt, dass wir alle von dieser Vermenschlichung der KI betroffen sind und dass es schneller passieren kann als wir denken. Du zeigst also mit dem Abschnitt "Ethische Folgen" nochmals klar auf, dass die Grenzen zwischen den Menschen und der KI immer mehr verwischen. 


\section{Verbesserungsvorschläge}
Die Quellen müssten in der bibliography abgelegt werden. Ich glaube du hast sie einfach im Paper am Schluss hinzugefügt. Beim nächsten Mal würde ich sie unter bibliography auflisten, damit sie vollständig erscheinen. Ich würde vielleicht noch ein paar Bilder einfügen um das Dokument noch ein wenig mehr zu gestalten. Vielleicht könnte man die Schriftart an einigen Stellen noch anpassen, damit das Dokument etwas lebendiger wird. Ausserdem würde ich nicht ein Kapitel mit Unterthemen machen, sondern verschiedene ganze Themen. Man könnte dem Aufatz noch mit Zitaten vervollständigen. Dies würde den Text noch ein bisschen hochwertiger machen. An sich finde ich den Aufsatz aber sehr gut und interessant geschrieben und mit guten Beispielen abgerundet. Durch den Aufsatz wird einem bewusst wie die Grenzen zwischen den Menschen und der KI verwischen. Somit bist du deiner Leitfrage gut nachgekommen und lässt den Leser über das Thema nachdenken. 

\end{document}
